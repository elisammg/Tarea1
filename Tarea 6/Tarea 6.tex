\documentclass{article}
\author{Elisa Monzon Godoy}
\title{Tarea No. 6}
\begin{document}
\maketitle
\section{Ejercicio 1.}
\begin{itemize}
\item\textit{Dado un numero n que pertenece a los numero naturales unitarios y que este sea un n sucesor de cero ($\sigma n(0) = n$).}
\item\textit{Y cero (0) el primer numero de los numeros naturales unitarios.}
\end{itemize}
\subsection{Suma de dos numeros naturales:}
Caso base: 
\begin{center}
$n + 0 = n$\\
$\sigma(n) + m = \sigma(n+m)$
\end{center}
Caso inductivo:
\begin{center}
$\sigma(\sigma(0) + \sigma(\sigma(0))$\\
$\sigma(\sigma(0) +\sigma(\sigma(0))) $\\
$\sigma(\sigma(0+\sigma(0))))$\\
$\sigma(\sigma(\sigma(\sigma(0))))$
\end{center}
\emph{De manera que:\\
$\sigma(n) = a$, $m = b$, $\sigma(n+m)= c$}
\subsection{Multiplicacion de dos numeros naturales}
Caso base: 
\begin{center}
$n * 0 =0$\\
$\sigma(n) * m = \sigma((n)*m) +m$\\
\end{center}
Caso inductivo:
\begin{center}
$(\sigma(0) * \sigma(\sigma(0))$\\
$\sigma(0) + \sigma(0) + \sigma(\sigma(0))veces...+\sigma(0)$\\
$\sigma(0) + [\sigma(0) + \sigma(\sigma(0))veces...+\sigma(0)] $\\
$\sigma(0)+ [\sigma(0) * (\sigma(0)] $\\
$\sigma(\sigma(0))$
\end{center}
\subsection{Mayor que para numeros naturales}
Caso base: 
\begin{itemize}
\item$\sigma(0) > 0$
\item$\sigma(\sigma(n))>0$
\end{itemize}
Caso inductivo:
\begin{itemize}
\item$\sigma(\sigma(0) > \sigma(0)$\\
$\sigma(0) > 0$\\
\item$\sigma(\sigma(n)) > n$\\
$\sigma(\sigma(\sigma(n))) > \sigma(n)$\\
$\sigma(\sigma(n))>n$
\end{itemize}
\section{Ejercicio 2.}
\subsection{Demostracion 1}
\begin{center}
$n+0=n$:\\
$\sigma(0) + \sigma(n)$\\
$\sigma (0 + n)$\\
$\sigma(n)$
\end{center}
\subsection{Demostracion 2}
\begin{center}
$n+m = m+n$:\\
$\sigma(\sigma((0)) + \sigma(\sigma(\sigma((0)))$ o \\
$\sigma(\sigma(\sigma((0))) + \sigma(\sigma((0))$ =\\
$\sigma (\sigma(\sigma(0)) + \sigma(0))$\\
$\sigma(\sigma(\sigma(\sigma(\sigma(0))))))$
\end{center}
\subsection{Demostracion 3}
\begin{center}
$n \otimes \sigma(\sigma(0)) = n \otimes n$:\\
$\sigma(n) \otimes \sigma(\sigma(0))$\\
$\sigma (\sigma(0) \otimes n)$\\
$\sigma(n) \otimes \sigma(n)$
\end{center}
\end{document}