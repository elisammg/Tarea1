\documentclass{article}

\author{\textbf{Elisa Monzon}}

\title{Tarea No. 1}

\begin{document}
\maketitle
\section{Public class Que hacer}
\begin{enumerate}
\item public string \emph{Nombre};
este atributo lo agregue ya que toda actividad debe estar nombrada y por lo mismo es de tipo \emph{string} para que puedan ingresarse letras.
\item public int \emph{Fecha};
el atributo fecha es necesario para establecer el dia que se planea realizar el que hacer por lo mismo es tipo int para que pueda ingresarla.
\item public string \emph{Lugar};
es necesario indicar el lugar para saber en donde se llevara a cabo el que hacer y para ello lo indique de tipo string para que con letras se puede agregar.
\item public string \emph{Utensilios};
este atributo decidi agregarlo ya que siempre se necesita de materiales para poder completar el que hacer y es de tipo string para que puedan ser nombrados
\item public boolean \emph{Estado (proceso o terminado)};
el ultimo atributo indica el estado en el que se encuentra el que hacer, ya sea en proceso o terminado, por ello es de tipo boolean siendo falso que este en proceso y verdadero cuando este terminado
\end{enumerate}
\section{Public class Que hacer}
\begin{enumerate}
\item Prioridad: esta funcion es importante para determinar cual quehacer es mas importante en comparacion de los demas.
\item Numerar: es necesario listarlos para ver el tamaño de la lista y poder establecer un limite en ella.
\item Agregar nuevas tareas ya que conforme los quehaceres se vayan generando,es necesario indicarlas en la lista
\item Eliminar un quehacer despues de que haya sido completado.
\item Buscar ya sea por orden alfabetico, prioridad o cualquier otro atributo del quehacer para encontrar el objeto deseado.

\end{enumerate}
\end{document}